\chapter{Parcial 1}


\section{Sobre esta guía}

La siguiente guía tiene como objetivo presentar de manera sistemática los contenidos teórico práctico de la asignatura \textbf{Programación de aplicaciones web} en base al plan de carrera del Instituto Tecnológico Benito Juárez en su rediseño del año 2016 para Desarrollo de Software.

\begin{center}
	Datos generales de la asignatura\\
		\begin{tabular}{ |c|c| } 
			\hline
			Nombre de la asignatura: & Programación Aplicaciones Web \\
			\hline
			Campo de formación: & Adaptación tecnológica e innovación \\ 
			\hline
			Unidad de organización curricular: & Formación técnica profesional \\
			\hline
			Número de período académico: & 4 \\
			\hline
			Número de horas de la asignatura: & 122 \\
			\hline
			Número de horas por cada componente: & 
			\begin{tabular}{c}
				Docencia: 60\\
				Prácticas de aprendizaje: 24 \\ 
				Aprendizaje autónomo: 38 \\
			\end{tabular} \\
			\hline
			Docente: & Freddy Heredia: \textbf{\texttt{fheredia@yavirac.edu.ec}}\\
			\hline
		\end{tabular}
\end{center}


\section{Lo que aprenderás}

\section{El origen}
https://rua.ua.es/dspace/bitstream/10045/16995/1/sergio_lujan-programacion_de_aplicaciones_web.pdf

\section{Arquitectura inicial}

Guerros de la red
https://youtu.be/1c2U1R8XXvA

\section{HTTP}
\subsection{Status}
\subsection{Métodos}
\subsection{URI - URL}
\subsection{Headers}

\section{Modelo de Objetos de Documento - DOM}

\section{Gestor de contenidos}

\section{REST}

\section{Entorno de desarrollo}

nodejs
https://github.com/nodesource/distributions/blob/master/README.md

Angular: npm install -g @angular/cli

Set-ExecutionPolicy RemoteSigned -Scope CurrentUser
get-ExecutionPolicy -list

\section{Backend}
\subsection{Postman - Thunder client - curl}
\subsection{Frameworks}
\subsection{Java}
\subsection{Spring boot}
\subsection{Spring data}
\subsection{Postgres - Mongo - Elastic Search}
\subsection{CRUD}
\subsubsection{Entidades - DTOs}
\subsubsection{Repositorios}
\subsubsection{Servicios}
\subsubsection{Controladores}

\section{Frontend}
\subsection{Frameworks}
\subsection{Angular}
\subsection{HTML}
\subsection{CSS}
\subsubsection{Frameworks CSS}
\subsubsection{Tailwind CSS}
\subsection{Javascript - ECMAScript}
\subsection{Typescript}
\subsection{CRUD}
\subsubsection{Entidades - DTOs}
\subsubsection{Servicios}
\subsubsection{Componentes}
\subsubsection{Input - Output}
\subsection{Temas}
\subsection{Menús}
\section{Ramificación GIT}